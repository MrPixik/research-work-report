\chapter{Локализация функционально неактивных областей}
\label{ch:chap2}
В данном разделе описывается последовательность этапов обработки ЭЭГ-сигналов, направленных на извлечение и интерпретацию независимых нейрональных источников. После предварительной очистки данных выполняется разложение сигналов с помощью независимого компонентного анализа (АНК), а затем для каждой полученной компоненты решается обратная задача, что позволяет локализовать соответствующие источники активности в коре головного мозга.
\section{Извлечение частотных диапазонов}

Как обсуждалось ранее, сигналы, записываемые в процессе ЭЭГ, представляют собой колебания потенциалов. Эти колебания могут быть классифицированы по частотным диапазонам, которые соответствуют различным состояниям мозга и когнитивным процессам. Основные частотные диапазоны, выделяемые в ЭЭГ-сигналах, включают:

\begin{enumerate}
    \item \textbf{Дельта-ритм:}
    Частотный диапазон от 0 до 4 Гц. Дельта-ритм связан с глубоким сном, состоянием покоя и восстановления. В норме он наблюдается у здоровых людей только в состоянии глубокого сна, но у пациентов с эпилепсией или органическими поражениями мозга может проявляться в состоянии бодрствования.
    \item \textbf{Тета-ритм:}
     Частотный диапазон от 4 до 8 Гц. Тета-ритм ассоциируется с состоянием расслабления, медитацией, сном и эмоциональными процессами. Он также связан с памятью и когнитивным функционированием, особенно в контексте релаксации и сна.
    \item \textbf{Альфа-ритм:}
    Частотный диапазон от 8 до 12 Гц. Альфа-ритм является наиболее изученным и связан с состоянием спокойного бодрствования, расслабления и отсутствия внешних стимулов. Он наиболее выражен в затылочных областях и уменьшается при возникновении внимания или активации.
    \item \textbf{Бета-ритм:}
    Частотный диапазон от 12 до 30 Гц. Бета-ритм связан с активным состоянием бодрствования, когнитивной нагрузкой, вниманием и обработкой информации. Он отражает активацию корковых зон, участвующих в высших когнитивных функциях.
    \item \textbf{Гамма-ритм:}
    Частотный диапазон от 30 до 45 Гц.. Гамма-ритм связан с высокоуровневыми когнитивными процессами, такими как восприятие, внимание, обработка сложной информации и интеграция сенсорных данных. В последние годы исследования указывают на важную роль гамма-активности в когнитивных и эмоциональных процессах.
\end{enumerate}

В контексте поставленной задачи основное внимание уделяется альфа- и бета-диапазонам ЭЭГ-активности. В связи с этим предварительно обработанные данные подвергаются дополнительной полосовой фильтрации, ограничивающей спектр сигналов соответствующими частотными диапазонами. Подробности реализации процедуры фильтрации был приведены в разделе \ref{sec:freq-filtr}.


\section{Разложение на независимые компоненты}
После предварительной обработки и удаления артефактов следующим этапом анализа является разложение ЭЭГ-сигналов на независимые компоненты с использованием метода анализа независимых компонент (АНК), математическое описание которого было подробно рассмотрено в разделе \ref{sec:ica}. Данная процедура применяется отдельно для каждого из предварительно выделенных частотных диапазонов и направлена на декомпозицию сигналов на пространственно стабильные, статистически независимые компоненты, отражающие различные источники нейрональной активности. Такой подход позволяет выявить функционально обособленные нейрональные процессы, проявляющиеся в соответствующих частотных диапазонах, и служит основой для последующей локализации источников этих компонентов в коре головного мозга.

\section{Решение обратной задачи}

После выделения независимых компонент из ЭЭГ-сигналов с помощью ICA, следующим этапом анализа является решение обратной задачи. Ее суть заключается в определении источников мозговой активности на основании сигналов, зарегистрированных с поверхности головы. Поскольку данная задача математически некорректна (существует бесконечное множество решений), применяется регуляризация, обеспечивающая устойчивость решения. В настоящем исследовании использован подход, основанный на минимизации нормы распределения источников, с последующим нормированием по оценке шума, что позволяет получить значения, пропорциональные вероятности активации каждого участка.

Для получения решения обратной задачи применяется анатомическая модель мозга и информация о расположении регистрирующих электродов. Такие данные позволяют сформировать оператор перехода от источников к наблюдаемым данным и, соответственно, использовать его для восстановления пространственного распределения активности.

Математическая формулировка прямой задачи ЭЭГ имеет вид:

\begin{equation}
\mathbf{X}(t) = \mathbf{G} \cdot \mathbf{J}(t) + \boldsymbol{\varepsilon}(t),
\label{eq:forward}
\end{equation}

где:
\begin{itemize}
    \item \(\mathbf{X}(t) \in \mathbb{R}^{M}\) — вектор потенциалов, измеренных на \(M\) электродах в момент времени \(t\),
    \item \(\mathbf{G} \in \mathbb{R}^{M \times N}\) — матрица перехода (forward model), описывающая вклад каждого источника в каждый электрод,
    \item \(\mathbf{J}(t) \in \mathbb{R}^{N}\) — вектор токов в \(N\) источниках,
    \item \(\boldsymbol{\varepsilon}(t)\) — вектор шума наблюдений.
\end{itemize}

Обратная задача заключается в оценке \(\mathbf{J}(t)\) по \(\mathbf{X}(t)\). Один из подходов — метод минимальной нормы:

\begin{equation}
\hat{\mathbf{J}}(t) = \mathbf{W} \cdot \mathbf{X}(t),
\label{eq:inverse}
\end{equation}

где:
\begin{itemize}
    \item \(\hat{\mathbf{J}}(t)\) — оценка распределения источников в момент времени \(t\),
    \item \(\mathbf{W}\) — инверсионный (обратный) оператор.
\end{itemize}

Инверсионный оператор \(\mathbf{W}\) может быть вычислен следующим образом:

\begin{equation}
\mathbf{W} = \mathbf{R} \cdot \mathbf{G}^T \cdot \left( \mathbf{G} \cdot \mathbf{R} \cdot \mathbf{G}^T + \lambda^2 \cdot \mathbf{C} \right)^{-1},
\label{eq:inverse_operator}
\end{equation}

где:
\begin{itemize}
    \item \(\mathbf{R}\) — априорная ковариационная матрица источников,
    \item \(\mathbf{C}\) — ковариационная матрица шума,
    \item \(\lambda^2 = \frac{1}{\text{SNR}^2}\) — параметр регуляризации, выраженный через отношение сигнал/шум (SNR).
\end{itemize}


Примененный алгоритм делится на следующие этапы:

\subsection{Получение инверсионного оператора}


Процесс расчёта инверсионного оператора включает несколько ключевых этапов:

\begin{enumerate}
    \item \textbf{Формирование модели головы и решения задачи распространения электрических потенциалов.} На этом этапе создаётся модель геометрии головы, учитывающая разные ткани (например, кору, жидкость и кости). Далее строится решение задачи распространения потенциалов от источников в мозгу к электродам на поверхности головы. Математически это может быть выражено как:

    \begin{equation}
    \mathbf{V} = \mathbf{G} \cdot \mathbf{S},
    \end{equation}
    где:
    \begin{itemize}
        \item \(\mathbf{V}\) — измеренные потенциалы на электродах,
        \item \(\mathbf{G}\) — матрица распространения, которая описывает, как активность каждого источника влияет на электродные сигналы,
        \item \(\mathbf{S}\) — вектор источников активности в мозге.
    \end{itemize}
    
    \item \textbf{Дискретизация возможных источников активности.} На этом этапе создаётся сетка точек, в которых могут находиться источники активности. Эти точки представляют собой априорные места активности в коре головного мозга. Для описания источников используется сетка с расположением \(N_{\text{src}}\) точек. Это позволяет представить распределение источников как вектор:

    \begin{equation}
    \mathbf{S} = \left[ S_1, S_2, \dots, S_{N_{\text{src}}} \right]^T,
    \end{equation}
    где каждый элемент \(S_i\) — это активность источника в точке \(i\).

    \item \textbf{Построение матрицы прямой задачи.} На основе геометрической модели и дискретных точек источников создаётся матрица, которая описывает, как активность каждого источника влияет на измеренные сигналы с каждого электрода. Эта матрица \(\mathbf{G}\) имеет размерность \([N_{\text{elec}} \times N_{\text{src}}]\), где \(N_{\text{elec}}\) — количество электродов, а \(N_{\text{src}}\) — количество точек источников.

    \begin{equation}
    \mathbf{G} = \begin{bmatrix}
    g_{11} & g_{12} & \dots & g_{1N_{\text{src}}} \\
    g_{21} & g_{22} & \dots & g_{2N_{\text{src}}} \\
    \vdots & \vdots & \ddots & \vdots \\
    g_{N_{\text{elec}}1} & g_{N_{\text{elec}}2} & \dots & g_{N_{\text{elec}}N_{\text{src}}}
    \end{bmatrix},
    \end{equation}
    где \(g_{ij}\) — это вклад источника \(S_j\) в измерение на электроде \(i\).

    \item \textbf{Формирование шумовой ковариации.} Здесь строится ковариационная матрица, которая описывает свойства шума в измерениях. Эта матрица имеет вид:

    \begin{equation}
    \mathbf{C} = \begin{bmatrix}
    \sigma_1^2 & \cdots & \cdots & 0 \\
    \vdots & \ddots & \vdots & \vdots \\
    0 & \cdots & \cdots & \sigma_{N_{\text{elec}}}^2
    \end{bmatrix},
    \end{equation}
    где \(\sigma_i^2\) — дисперсия шума на \(i\)-м электроде.

    \item \textbf{Построение инверсионного оператора.} На основе информации о прямой задаче и ковариации шума рассчитывается инверсионный оператор, который позволит восстановить источники активности на основе наблюдаемых сигналов. Формула для инверсионного оператора, как уже было представлено ранее, имеет вид:

    \begin{equation}
    \mathbf{W} = \mathbf{R} \cdot \mathbf{G}^T \cdot \left( \mathbf{G} \cdot \mathbf{R} \cdot \mathbf{G}^T + \lambda^2 \cdot \mathbf{C} \right)^{-1},
    \end{equation}
    где:
    \begin{itemize}
        \item \(\mathbf{W}\) — инверсионный оператор,
        \item \(\mathbf{G}\) — матрица прямой задачи,
        \item \(\mathbf{R}\) — ковариационная матрица источников,
        \item \(\mathbf{C}\) — ковариационная матрица шума,
        \item \(\lambda^2 = \frac{1}{\text{SNR}^2}\) — параметр регуляризации.
    \end{itemize}
\end{enumerate}

После получения инверсионного оператора он сохраняется для дальнейшего использования в процессе локализации активности источников. Этот оператор играет ключевую роль в обратной задаче, позволяя преобразовывать наблюдаемые данные в распределение активности по источникам в мозге.


\subsection{Формирование сигналов независимых компонент}

Временной сигнал \(s_k(t)\) каждой независимой компоненты \(k\) восстанавливается из исходных каналов следующим образом:

\begin{equation}
s_k(t) = \sum_{i=1}^{M} a_{ik} \cdot x_i(t),
\label{eq:ica_signal}
\end{equation}

где:
\begin{itemize}
    \item \(s_k(t)\) — временной сигнал \(k\)-й независимой компоненты,
    \item \(a_{ik}\) — коэффициент вклада \(i\)-го канала в \(k\)-ю компоненту (из матрицы ICA-разложения),
    \item \(x_i(t)\) — сигнал на \(i\)-м канале в момент времени \(t\),
    \item \(M\) — число каналов ЭЭГ.
\end{itemize}

\subsection{Сегментация сигнала на интервалы}

Временной ряд компоненты разбивается на сегменты фиксированной длины:

\begin{equation}
\text{Epoch}_j = s_k(t_j : t_{j+1}), \quad j = 1, \dots, P,
\label{eq:segmentation}
\end{equation}

где:
\begin{itemize}
    \item \(\text{Epoch}_j\) — \(j\)-й сегмент сигнала,
    \item \(t_j, t_{j+1}\) — начальный и конечный моменты времени для сегмента,
    \item \(P\) — общее количество сегментов.
\end{itemize}

\subsection{Решение обратной задачи для каждого сегмента}

Для каждого сегмента \(\text{Epoch}_j\) вычисляется пространственное распределение источников:

\begin{equation}
\hat{\mathbf{J}}_j = \mathbf{W} \cdot \text{Epoch}_j,
\label{eq:epoch_inverse}
\end{equation}

где:
\begin{itemize}
    \item \(\hat{\mathbf{J}}_j \in \mathbb{R}^{N \times T}\) — матрица активности \(N\) источников по \(T\) временным точкам,
    \item \(\mathbf{W} \in \mathbb{R}^{N \times M}\) — инверсионный оператор,
    \item \(\text{Epoch}_j \in \mathbb{R}^{M \times T}\) — матрица сегмента сигнала на \(M\) каналах.
\end{itemize}

\subsection{Усреднение активности по времени}

Для каждого источника \(n\) вычисляется средняя активность по времени:

\begin{equation}
\bar{J}_n = \frac{1}{T} \sum_{t=1}^{T} \hat{J}_n(t),
\label{eq:mean_activity}
\end{equation}

где:
\begin{itemize}
    \item \(\bar{J}_n\) — средняя активность \(n\)-го источника,
    \item \(\hat{J}_n(t)\) — активность источника \(n\) в момент времени \(t\),
    \item \(T\) — количество временных точек в сегменте.
\end{itemize}




\endinput