\chapter{Представление результатов}
Анализ результатов локализации источников мозговой активности осуществляется на основе визуализации инвертированных топографических карт, полученных из ЭЭГ данных для двух различных экспериментальных условий. Визуализация производится как в виде пространственного распределения активности на поверхности коры головного мозга, так и в проекции на анатомические области.

Визуальные карты активности строятся по результатам решения обратной задачи для каждой независимой компоненты, извлечённой из предварительно обработанных сигналов. Эти карты отражают локализацию предполагаемых нейронных источников, ответственых за генерацию наблюдаемых паттернов ЭЭГ-активности. Инвертированные карты представлены в проекции на стандартную мозговую модель, что обеспечивает сравнимость данных между субъектами и экспериментами.

Для более детализированного анализа пространственное распределение активности дополнительно декомпозируется по анатомическим областям мозга, с опорой на Бродмановскую классификацию коры. Каждой компоненте сопоставляется набор Бродмановских полей, в которых наблюдается наибольшая выраженность источниковой активности. Такая декомпозиция позволяет интерпретировать полученные данные в терминах функциональной организации коры головного мозга.

Кроме того, компоненты анализируются в контексте их частотной принадлежности, благодаря предварительной фильтрации сигналов по заданным диапазонам частот, что, в свою очередь, даёт основания для гипотез о функциональном значении наблюдаемых паттернов.

На основании полученных распределений будет проведена количественная оценка доли вклада каждого из Бродмановских полей в генерацию независимых компонентов. Это позволит выявить специфические области, наиболее активные при различных экспериментальных условиях и в различных частотных диапазонах, а также провести сравнение между условиями (например, «музыка» против «тишины»), что может пролить свет на механизмы нейронной активности в этих состояниях.

Таким образом, представленные результаты сочетают в себе как качественную визуальную интерпретацию активности, так и количественную статистическую оценку пространственного распределения нейронных источников в анатомическом и функциональном контексте.

\endinput