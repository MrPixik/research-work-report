\documentclass[a4paper,14pt,oneside,openany]{memoir}

%%% Задаем поля, отступы и межстрочный интервал %%%

\usepackage[left=30mm, right=15mm, top=20mm, bottom=20mm]{geometry} % Пакет geometry с аргументами для определения полей
\pagestyle{plain} % Убираем стандарные для данного класса верхние колонтитулы с заголовком текущей главы, оставляем только номер страницы снизу по центру
\parindent=1.25cm % Абзацный отступ 1.25 см, приблизительно равно пяти знакам, как по ГОСТ
\usepackage{indentfirst} % Добавляем отступ к первому абзацу
%\linespread{1.3} % Межстрочный интервал (наиболее близко к вордовскому полуторному) - тут вместо этого используется команда OnehalfSpacing*

%%% Задаем языковые параметры и шрифт %%%

\usepackage[english, russian]{babel}   % Настройки для русского языка как основного в тексте
\usepackage{hyphenat}
\babelfont{rm}{Times New Roman}                     % TMR в качестве базового roman-щрифта
\sloppy
\hyphenpenalty=10000
\exhyphenpenalty=10000
%%% Задаем стиль заголовков и подзаголовков в тексте %%%

\setsecnumdepth{subsection} % Номера азделов считать до третьего уровня включительно, т.е. нумеруются только главы, секции, подсекции
\renewcommand*{\chapterheadstart}{} % Переопределяем команду, задающую отступ над заголовком, чтобы отступа не было
\renewcommand*{\printchaptername}{} % Переопределяем команду, печатающую слово "Глава", чтобы оно не печалось
%\renewcommand*{\printchapternum}{} % То же самое для номера главы - тут не надо, номер главы оставляем
\renewcommand*{\chapnumfont}{\normalfont\bfseries} % Меняем стиль шрифта для номера главы: нормальный размер, полужирный
\renewcommand*{\afterchapternum}{\hspace{1em}} % Меняем разделитель между номером главы и названием
\renewcommand*{\printchaptertitle}{\normalfont\bfseries\centering\MakeUppercase} % Меняем стиль написания для заголовка главы: нормальный размер, полужирный, центрированный, заглавными буквами
\setbeforesecskip{20pt} % Задаем отступ перед заголовком секции
\setaftersecskip{20pt} % Ставим такой же отступ после заголовка секции
\setsecheadstyle{\raggedright\normalfont\bfseries} % Меняем стиль написания для заголовка секции: выравнивание по правому краю без переносов, нормальный размер, полужирный
\setbeforesubsecskip{20pt} % Задаем отступ перед заголовком подсекции
\setaftersubsecskip{20pt} % Ставим такой же отступ после заголовка подсекции
\setsubsecheadstyle{\raggedright\normalfont\bfseries}  % Меняем стиль написания для заголовка подсекции: выравнивание по правому краю без переносов, нормальный размер, полужирный

%%% Задаем параметры оглавления %%%

\addto\captionsrussian{\renewcommand\contentsname{Содержание}} % Меняем слово "Оглавление" на "Содержание"
\setrmarg{2.55em plus1fil} % Запрещаем переносы слов в оглавлении
%\setlength{\cftbeforechapterskip}{0pt} % Эта команда убирает интервал между заголовками глав - тут не надо, так красивее смотрится
\renewcommand{\aftertoctitle}{\afterchaptertitle \vspace{-\cftbeforechapterskip}} % Делаем отступ между словом "Содержание" и первой строкой таким же, как у заголовков глав
%\renewcommand*{\chapternumberline}[1]{} % Делаем так, чтобы номер главы не печатался - тут не надо
\renewcommand*{\cftchapternumwidth}{1.5em} % Ставим подходящий по размеру разделитель между номером главы и самим заголовком
\renewcommand*{\cftchapterfont}{\normalfont\MakeUppercase} % Названия глав обычным шрифтом заглавными буквами
\renewcommand*{\cftchapterpagefont}{\normalfont} % Номера страниц обычным шрифтом
\renewcommand*{\cftchapterdotsep}{\cftdotsep} % Делаем точки до номера страницы после названий глав
\renewcommand*{\cftdotsep}{1} % Задаем расстояние между точками
\renewcommand*{\cftchapterleader}{\cftdotfill{\cftchapterdotsep}} % Делаем точки стандартной формы (по умолчанию они "жирные")
\maxtocdepth{subsection} % В оглавление попадают только разделы первыхтрех уровней: главы, секции и подсекции

%%% Выравнивание и переносы %%%

%% http://tex.stackexchange.com/questions/241343/what-is-the-meaning-of-fussy-sloppy-emergencystretch-tolerance-hbadness
%% http://www.latex-community.org/forum/viewtopic.php?p=70342#p70342
\tolerance 1414
\hbadness 1414
\emergencystretch 1.5em                             % В случае проблем регулировать в первую очередь
\hfuzz 0.3pt
\vfuzz \hfuzz
%\dbottom
%\sloppy                                            % Избавляемся от переполнений
\clubpenalty=10000                                  % Запрещаем разрыв страницы после первой строки абзаца
\widowpenalty=10000                                 % Запрещаем разрыв страницы после последней строки абзаца
\brokenpenalty=4991                                 % Ограничение на разрыв страницы, если строка заканчивается переносом

%%% Объясняем компилятору, какие буквы русского алфавита можно использовать в перечислениях (подрисунках и нумерованных списках) %%%
%%% По ГОСТ нельзя использовать буквы ё, з, й, о, ч, ь, ы, ъ %%%
%%% Здесь также переопределены заглавные буквы, хотя в принципе они в документе не используются %%%

\makeatletter
    \def\russian@Alph#1{\ifcase#1\or
       А\or Б\or В\or Г\or Д\or Е\or Ж\or
       И\or К\or Л\or М\or Н\or
       П\or Р\or С\or Т\or У\or Ф\or Х\or
       Ц\or Ш\or Щ\or Э\or Ю\or Я\else\xpg@ill@value{#1}{russian@Alph}\fi}
    \def\russian@alph#1{\ifcase#1\or
       а\or б\or в\or г\or д\or е\or ж\or
       и\or к\or л\or м\or н\or
       п\or р\or с\or т\or у\or ф\or х\or
       ц\or ш\or щ\or э\or ю\or я\else\xpg@ill@value{#1}{russian@alph}\fi}
\makeatother

%%% Задаем параметры оформления рисунков и таблиц %%%

\usepackage{graphicx, caption, subcaption} % Подгружаем пакеты для работы с графикой и настройки подписей
\graphicspath{{images/}} % Определяем папку с рисунками
\captionsetup[figure]{font=small, width=\textwidth, name=Рисунок, justification=centering} % Задаем параметры подписей к рисункам: маленький шрифт (в данном случае 12pt), ширина равна ширине текста, полнотекстовая надпись "Рисунок", выравнивание по центру
\captionsetup[subfigure]{font=small} % Индексы подрисунков а), б) и так далее тоже шрифтом 12pt (по умолчанию делает еще меньше)
\captionsetup[table]{singlelinecheck=false,font=small,width=\textwidth,justification=justified} % Задаем параметры подписей к таблицам: запрещаем переносы, маленький шрифт (в данном случае 12pt), ширина равна ширине текста, выравнивание по ширине
\captiondelim{ --- } % Разделителем между номером рисунка/таблицы и текстом в подписи является длинное тире
\setkeys{Gin}{width=\textwidth} % По умолчанию размер всех добавляемых рисунков будет подгоняться под ширину текста
\renewcommand{\thesubfigure}{\asbuk{subfigure}} % Нумерация подрисунков строчными буквами кириллицы
%\setlength{\abovecaptionskip}{0pt} % Отбивка над подписью - тут не меняем
%\setlength{\belowcaptionskip}{0pt} % Отбивка под подписью - тут не меняем
\usepackage[section]{placeins} % Объекты типа float (рисунки/таблицы) не вылезают за границы секциии, в которой они объявлены

%%% Задаем параметры ссылок и гиперссылок %%% 

\usepackage{hyperref}                               % Подгружаем нужный пакет
\hypersetup{
    colorlinks=true,                                % Все ссылки и гиперссылки цветные
    linktoc=all,                                    % В оглавлении ссылки подключатся для всех отображаемых уровней
    linktocpage=true,                               % Ссылка - только номер страницы, а не весь заголовок (так выглядит аккуратнее)
    linkcolor=red,                                  % Цвет ссылок и гиперссылок - красный
    citecolor=red                                   % Цвет цитировний - красный
}

%%% Настраиваем отображение списков %%%

\usepackage{enumitem}                               % Подгружаем пакет для гибкой настройки списков
\renewcommand*{\labelitemi}{\normalfont{--}}        % В ненумерованных списках для пунктов используем короткое тире
\makeatletter
    \AddEnumerateCounter{\asbuk}{\russian@alph}     % Объясняем пакету enumitem, как использовать asbuk
\makeatother
\renewcommand{\labelenumii}{\asbuk{enumii})}        % Кириллица для второго уровня нумерации
\renewcommand{\labelenumiii}{\arabic{enumiii})}     % Арабские цифры для третьего уровня нумерации
\setlist{noitemsep, leftmargin=*}                   % Убираем интервалы между пунками одного уровня в списке
\setlist[1]{labelindent=\parindent}                 % Отступ у пунктов списка равен абзацному отступу
\setlist[2]{leftmargin=\parindent}                  % Плюс еще один такой же отступ для следующего уровня
\setlist[3]{leftmargin=\parindent}                  % И еще один для третьего уровня

%%% Счетчики для нумерации объектов %%%

\counterwithout{figure}{chapter}                    % Сквозная нумерация рисунков по документу
\counterwithout{equation}{chapter}                  % Сквозная нумерация математических выражений по документу
\counterwithout{table}{chapter}                     % Сквозная нумерация таблиц по документу

%%% Реализация библиографии пакетами biblatex и biblatex-gost с использованием движка biber %%%

\usepackage{csquotes} % Пакет для оформления сложных блоков цитирования (biblatex рекомендует его подключать)
\usepackage[%
backend=biber,                                      % Движок
bibencoding=utf8,                                   % Кодировка bib-файла
sorting=none,                                       % Настройка сортировки списка литературы
style=gost-numeric,                                 % Стиль цитирования и библиографии по ГОСТ
language=auto,                                      % Язык для каждой библиографической записи задается отдельно
autolang=other,                                     % Поддержка многоязычной библиографии
sortcites=true,                                     % Если в квадратных скобках несколько ссылок, то отображаться будут отсортированно
movenames=false,                                    % Не перемещать имена, они всегда в начале библиографической записи
maxnames=5,                                         % Максимальное отображаемое число авторов
minnames=3,                                         % До скольки сокращать число авторов, если их больше максимума
doi=false,                                          % Не отображать ссылки на DOI
isbn=false,                                         % Не показывать ISBN, ISSN, ISRN
]{biblatex}[2016/09/17]
\DeclareDelimFormat{bibinitdelim}{}                 % Убираем пробел между инициалами (Иванов И.И. вместо Иванов И. И.)
\addbibresource{biba.bib}                           % Определяем файл с библиографией

%%% Скрипт, который автоматически подбирает язык (и, следовательно, формат) для каждой библиографической записи %%%
%%% Если в названии работы есть кириллица - меняем значение поля langid на russian %%%
%%% Все оставшиеся пустые места в поле langid заменяем на english %%%

\DeclareSourcemap{
  \maps[datatype=bibtex]{
    \map{
        \step[fieldsource=title, match=\regexp{^\P{Cyrillic}*\p{Cyrillic}.*}, final]
        \step[fieldset=langid, fieldvalue={russian}]
    }
    \map{
        \step[fieldset=langid, fieldvalue={english}]
    }
  }
}

%%% Прочие пакеты для расширения функционала %%%

\usepackage{longtable,ltcaption}                    % Длинные таблицы
\usepackage{multirow,makecell}                      % Улучшенное форматирование таблиц
\usepackage{booktabs}                               % Еще один пакет для красивых таблиц
\usepackage{soul}                                   % Поддержка переносоустойчивых подчёркиваний и зачёркиваний
\usepackage{icomma}                                 % Запятая в десятичных дробях
\usepackage{hyphenat}                               % Для красивых переносов
\usepackage{textcomp}                               % Поддержка "сложных" печатных символов типа значков иены, копирайта и т.д.
\usepackage[version=4]{mhchem}                      % Красивые химические уравнения
\usepackage{amsmath, amsfonts, amssymb}                                % Усовершенствование отображения математических выражений 
\usepackage{media9}

%%% Вставляем по очереди все содержательные части документа %%%

\begin{document}

\thispagestyle{empty}

\begin{center}
    МИНИСТЕРСТВО НАУКИ И ВЫСШЕГО ОБРАЗОВАНИЯ \\ РОССИЙСКОЙ ФЕДЕРАЦИИ

    \vspace{20pt}

    Федеральное государственное автономное \\ образовательное учреждение высшего образования \\
    «Национальный исследовательский ядерный университет “МИФИ”» 

    \vspace{20pt}

    {Научно-образовательный центр НЕВОД}
\end{center}

\vfill

\begin{center}
    ОТЧЕТ \\  
     о прохождении производственной практики \\ (научно-исследовательской работы)

    \vspace{20pt}

    \uppercase{«Разработка новых алгоритмов для локализации отдельных зон коры головного мозга»}
\end{center}

\vfill
\hfill
\begin{flushright}
    \begin{tabular}{rl}
        Студент: & Чернышев С.А. \\[0.5cm]
        Группа: & Б21-215 \\[0.5cm]
        Научный руководитель: & Климанов С.Г. \\
    \end{tabular}
\end{flushright}
\vfill

\begin{center}
    г. Москва\\
 2024
\end{center}                                     % Титульник

\newpage % Переходим на новую страницу
\setcounter{page}{2} % Начинаем считать номера страниц со второй
\OnehalfSpacing* % Задаем полуторный интервал текста (в титульнике одинарный, поэтому команда стоит после него)


\tableofcontents*                                   % Автособираемое оглавление
\chapter*{Аннотация}
\label{ch:annotation}
В данной работе рассматривается задача локализации зон коры головного мозга с пониженной нейронной активностью в условиях двух различных когнитивных состояний: при прослушивании музыкального фрагмента и в тишине. Целью исследования является разработка алгоритма, позволяющего выявлять и визуализировать участки мозга, демонстрирующие минимальную активность в каждом из условий. Предложенный алгоритм включает в себя этапы предобработки данных ЭЭГ (частотная фильтрация, удаление артефактов с использованием метода независимого компонентного анализа и автоматической классификации ICAlabel), повторную декомпозицию сигналов по частотным диапазонам, а также решение обратной задачи для пространственной локализации активности.
Особенностью подхода является ориентация на зоны с наименьшей, а не максимальной, активностью, что позволяет исследовать механизмы подавления или «отключения» отдельных участков коры в зависимости от сенсорного контекста.
\vspace*{-\baselineskip}
  
\newpage

\chapter*{Введение}
\addcontentsline{toc}{chapter}{Введение}
\label{ch:intro}
Электроэнцефалограмма (ЭЭГ) — метод регистрации электрической активности головного мозга, который применяется для диагностики неврологических расстройств, мониторинга функционального состояния мозга и научных исследований в области нейрофизиологии. В медицинской практике она используется для диагностики эпилепсии, расстройств сна, оценивания состояния мозга после травм, инсультов и других паталогических состояний. В нейрофизиологии и когнитивных науках ЭЭГ применяется для изучения функций мозга, таких как внимание, восприятие и память.

ЭЭГ позволяет получить информацию о временной динамике и распределении электрической активности мозга с помощью электродов, размещенных на поверхности головы. Сигналы, записываемые в процессе ЭЭГ, представляют собой колебания потенциалов, возникающих в результате активности нейронных связей. В классических парадигмах анализа ЭЭГ основное внимание уделяется активным зонам — тем участкам коры, где наблюдается повышение мощности в определённых диапазонах частот. Предполагается, что именно эти зоны вовлечены в текущую когнитивную или сенсорную задачу. Однако в настоящей работе ставится под сомнение эта традиционная интерпретация. Исходной гипотезой исследования является предположение о том, что наиболее важные для текущей задачи участки мозга могут проявляться на ЭЭГ не как активные, а наоборот — как участки с пониженной выраженностью сигналов, то есть как "неактивные зоны" в классическом смысле. Такая интерпретация может отражать усиление локального торможения, перераспределение внимания или подавление фона, и, следовательно, быть маркером функционального вовлечения.

Целью данной работы является разработка алгоритма для локализации участков коры головного мозга, демонстрирующих пониженную активность, и последующее сравнение этих участков в двух различных условиях: при прослушивании музыки и в состоянии тишины. Таким образом, работа направлена на поиск тех зон мозга, которые наиболее вовлечены в восприятие музыки или в пассивное состояние, вопреки традиционному акценту на максимумах ЭЭГ-сигнала.
Для проведения исследования использовались данные электроэнцефалографии, зарегистрированные у группы из 27 испытуемых в двух экспериментальных условиях: в состоянии акустического покоя (тишины) и при прослушивании аудиостимулов. Сигналы ЭЭГ регистрировались с использованием 128-канальной системы, обеспечивающей высокое пространственное разрешение. Продолжительность каждого индивидуального измерения составляла приблизительно 180 секунд.

Для достижения этой цели были поставлены следующие задачи:

\begin{enumerate}
    \item провести предварительную обработку ЭЭГ-данных, включая частотную фильтрацию и автоматическое удаление артефактов с помощью ICAlabel.
    \item выделить конкретные частотные диапазоны и провести повторное разложение на независимые компоненты (ICA).
    \item решить обратную задачу ЭЭГ, с акцентом на компоненты, демонстрирующие минимальную активность.
    \item визуализировать и проанализировать полученные карты неактивности для двух наборов данных.
\end{enumerate}

Настоящая работа включает в себя теоретическое обоснование применяемых методов, описание алгоритма обработки и анализа ЭЭГ-данных, а также обсуждение полученных результатов в контексте альтернативной интерпретации нейронной активности.

\endinput

%Электроэнцефалограмма (ЭЭГ) — метод регистрации электрической активности головного мозга, который применяется для диагностики неврологических расстройств, мониторинга функционального состояния мозга и научных исследований в области нейрофизиологии. В медицинской практике она используется для диагностики эпилепсии, расстройств сна, оценивания состояния мозга после травм, инсультов и других паталогических состояний. В нейрофизиологии и когнитивных науках ЭЭГ применяется для изучения функций мозга, таких как внимание, восприятие и память.

% ЭЭГ позволяет получить информацию о временной динамике и распределении электрической активности мозга с помощью электродов, размещенных на поверхности головы. Сигналы, записываемые в процессе ЭЭГ, представляют собой колебания потенциалов, возникающих в результате активности нейронных связей. Эти колебания могут быть классифицированы по частотным диапазонам, которые соответствуют различным состояниям мозга и когнитивным процессам. Основные частотные диапазоны, выделяемые в ЭЭГ-сигналах, включают:
% \begin{enumerate}
%     \item Дельта-ритм (Delta):
%     Частотный диапазон от 0 до 4 Гц. Дельта-ритм связан с глубоким сном, состоянием покоя и восстановления. В норме он наблюдается у здоровых людей только в состоянии глубокого сна, но у пациентов с эпилепсией или органическими поражениями мозга может проявляться в состоянии бодрствования.
%     \item Тета-ритм (Theta):
%      Частотный диапазон от 4 до 8 Гц. Тета-ритм ассоциируется с состоянием расслабления, медитацией, сном и эмоциональными процессами. Он также связан с памятью и когнитивным функционированием, особенно в контексте релаксации и сна.
%     \item Альфа-ритм (Alpha):
%     Частотный диапазон от 8 до 12 Гц. Альфа-ритм является наиболее изученным и связан с состоянием спокойного бодрствования, расслабления и отсутствия внешних стимулов. Он наиболее выражен в затылочных областях и уменьшается при возникновении внимания или активации.
%     \item Бета-ритм (Beta):
%     Частотный диапазон от 12 до 30 Гц. Бета-ритм связан с активным состоянием бодрствования, когнитивной нагрузкой, вниманием и обработкой информации. Он отражает активацию корковых зон, участвующих в высших когнитивных функциях.
%     \item Гамма-ритм (Gamma):
%     Частотный диапазон от 30 до 45 Гц.. Гамма-ритм связан с высокоуровневыми когнитивными процессами, такими как восприятие, внимание, обработка сложной информации и интеграция сенсорных данных. В последние годы исследования указывают на важную роль гамма-активности в когнитивных и эмоциональных процессах.
% \end{enumerate}
% Однако, кроме полезных сигналов, в данные ЭЭГ попадают различные помехи, которые необходимо устранить для корректного анализа. Одним из наиболее часто встречающихся типов помех являются низкочастотные дрейфы, которые обычно имеют частоту ниже 1 Гц и могут быть вызваны неравномерным дыханием, изменениями кровотока или артериального давления, а также изменениями температуры тела. Эти помехи проявляются в виде медленных изменений базовой линии сигнала. 
% Другой распространенный тип помех — мышечные артефакты, обусловленные активностью мышц головы, таких как жевательные мышцы или движения глаз, которые накладывают высокочастотные шумы на ЭЭГ-сигналы.

% После предварительной очистки и выделения интересующих частотных диапазонов, проводится разложение сигналов методом ICA. Полученные компоненты далее используются для пространственной локализации источников мозговой активности посредством решения обратной задачи ЭЭГ.

% Решение обратной задачи направлено на определение пространственного распределения активности в головном мозге по данным с поверхности черепа. Это фундаментально некорректная (недоопределённая) задача, однако современные методы, такие как LORETA, sLORETA или другие линейные модели, позволяют получить приближенные оценки. В данной работе решается инвертированная обратная задача, где целью является не выявление активных зон, а, наоборот, поиск участков, демонстрирующих наименьшую активность. Для этого рассчитывается среднее значение сигнала по каналам, после чего из максимального значения по каждому каналу вычитается это среднее. Такой подход позволяет визуализировать области, систематически демонстрирующие подавленную активность, что может быть важным маркером состояний подавления, «отключения» или перераспределения внимания.

% Выбор метода обратной задачи обусловлен необходимостью не просто анализировать временные характеристики ЭЭГ, но и получить топографическую карту корковых источников, что критично при сравнении функциональной архитектуры мозга между различными экспериментальными условиями.                                     % Введение
\chapter{Предобработка ЭЭГ-данных}
\label{ch:chap1}
Предобработка ЭЭГ данных критически важна для дальнейшего анализа и интерпретации результатов. Она включает в себя несколько ключевых этапов, направленных на улучшение качества сигналов и снижение влияния помех. В настоящей главе представлен полный алгоритм предобработки данных, который был применен в данной работе, а также краткий теоретический экскурс по каждому из используемых методов.

\section{Частотная фильтрация}
\label{sec:freq-filtr}
ЭЭГ-сигналы подвержены различным высокочастотным и низкочастотным помехам, таким как шумы от электроприборов, мышечных артефактов и дыхания. Для удаления нежелательных частот применяют фильтрацию.

В частности, в данной программе была примененен высокочастотный КИХ-фильтр, построенный на основе окна Хэмминга, для удаления низкочастотных дрейфов.

В данном фильтре фильтрация временного ряда выполняется с помощью свёртки исходного сигнала \(x[n]\) с импульсной характеристикой фильтра \(h[k]\). Формула фильтрации имеет вид:

\begin{equation}
y[n] = \sum_{k=0}^{M} h[k] \cdot x[n-k], \quad n = M, \dots, N-M,
\label{eq:fir_filter}
\end{equation}

где:
\begin{itemize}
    \item \(y[n]\) — выходной (отфильтрованный) сигнал,
    \item \(x[n]\) — входной сигнал,
    \item \(h[k]\) — импульсная характеристика фильтра,
    \item \(M\) — длина фильтра,
    \item \(N\) — длина входного сигнала.
\end{itemize}

Для сглаживания коэффициентов фильтра используется окно Хэмминга, определяемое следующим образом:

\begin{equation}
w[k] = 0.54 - 0.46 \cos\left( \frac{2 \pi k}{M} \right), \quad k = 0, \dots, M.
\label{eq:hamming_window}
\end{equation}

Импульсная характеристика высокочастотного фильтра с частотой отсечки \(f_c\) задаётся как:

\begin{equation}
h[k] = w[k] \cdot \left( \delta[k] - \frac{\sin(2 \pi f_c (k - M/2))}{\pi (k - M/2)} \right), \quad k = 0, \dots, M,
\label{eq:filter_kernel}
\end{equation}

где:
\begin{itemize}
    \item \(\delta[k]\) — дельта-функция (единица при \(k = M/2\), иначе ноль),
    \item \(f_c\) — частота отсечки в долях от частоты дискретизации,
    \item \(w[k]\) — окно Хэмминга (см. формулу \eqref{eq:hamming_window}).
\end{itemize}

Выбор именно этого фильтра объясняется следующими соображениями. Во-первых, он не искажает форму сигнала, что важно для ЭЭГ, где даже небольшие сдвиги могут исказить результаты анализа. Во-вторых, окно Хэмминга помогает отсечь частоты вне нашего диапазона без сильного «размытия» полезной части сигнала. Также, применение КИХ-фильтров для предварительной фильтрации ЭЭГ является общепринятым подходом.

\section{Удаление мимических артефактов.}
\label{sec:ica}
ЭЭГ данные помимо высокочастотных и нихкочастотных помех содержат различные мимческие артефакты, частоты которых совпадают с частотами, которые несут полезную информацию о мозговой активности. Для удаленяия такого рода артефактов, данные предварительно разбиваются на компоненты посредством применения метода анализа независимых компонент (АНК). 

Математически модель АНК можно представить следующим образом:

\begin{equation}
\mathbf{X} = \mathbf{A} \mathbf{S},
\label{eq:ica_model}
\end{equation}

где:
\begin{itemize}
    \item \(\mathbf{X} \in \mathbb{R}^{n \times T}\) — матрица наблюдаемых сигналов ЭЭГ, где \(n\) — количество каналов, \(T\) — количество временных отсчётов;
    \item \(\mathbf{A} \in \mathbb{R}^{n \times n}\) — неизвестная матрица смешивания, описывающая вклад каждой компоненты в каждый канал;
    \item \(\mathbf{S} \in \mathbb{R}^{n \times T}\) — матрица скрытых независимых компонент, которые необходимо восстановить.
\end{itemize}

Цель АНК заключается в нахождении такой матрицы \(\mathbf{W}\), которая аппроксимирует обратную матрицу смешивания \(\mathbf{A}^{-1}\) и позволяет выделить независимые компоненты:

\begin{equation}
\mathbf{S} = \mathbf{W} \mathbf{X},
\label{eq:ica_unmixing}
\end{equation}

где:
\begin{itemize}
    \item \(\mathbf{W} \in \mathbb{R}^{n \times n}\) — матрица разведения (unmixing matrix), обучаемая на основе статистической независимости компонент;
    \item \(\mathbf{S}\) — восстановленные независимые источники сигнала.
\end{itemize}

Метод АНК основывается на предположении, что все компоненты \(\mathbf{S}\) являются статистически независимыми и неггауссовыми (за исключением, возможно, одной). Благодаря этому он эффективно разделяет смешанные сигналы и позволяет локализовать артефакты, не прибегая к априорной информации о природе источников.

После разложения сигналов на компоненты с помощью АНК производится автоматическая классификация компонент с использованием инструмента ICLabel. Он представляет собой обученную модель, которая относит полученные компоненты к различным классам (мозг, мышцы, глаза, шум и т.д.). После классификации, нежелательные компоненты удаляются.

Использование АНК в сочетании с ICLabel обеспечивает эффективное и автоматизированное удаление артефактов из ЭЭГ-данных, сохраняя при этом значимую нейронную информацию. Этот подход широко применяется в нейрофизиологических исследованиях и клинической практике для повышения качества анализа мозговой активности.

\endinput                                     % Первая глава
\chapter{Локализация функционально неактивных областей}
\label{ch:chap2}
В данном разделе описывается последовательность этапов обработки ЭЭГ-сигналов, направленных на извлечение и интерпретацию независимых нейрональных источников. После предварительной очистки данных выполняется разложение сигналов с помощью независимого компонентного анализа (АНК), а затем для каждой полученной компоненты решается обратная задача, что позволяет локализовать соответствующие источники активности в коре головного мозга.
\section{Извлечение частотных диапазонов}

Как обсуждалось ранее, сигналы, записываемые в процессе ЭЭГ, представляют собой колебания потенциалов. Эти колебания могут быть классифицированы по частотным диапазонам, которые соответствуют различным состояниям мозга и когнитивным процессам. Основные частотные диапазоны, выделяемые в ЭЭГ-сигналах, включают:

\begin{enumerate}
    \item \textbf{Дельта-ритм:}
    Частотный диапазон от 0 до 4 Гц. Дельта-ритм связан с глубоким сном, состоянием покоя и восстановления. В норме он наблюдается у здоровых людей только в состоянии глубокого сна, но у пациентов с эпилепсией или органическими поражениями мозга может проявляться в состоянии бодрствования.
    \item \textbf{Тета-ритм:}
     Частотный диапазон от 4 до 8 Гц. Тета-ритм ассоциируется с состоянием расслабления, медитацией, сном и эмоциональными процессами. Он также связан с памятью и когнитивным функционированием, особенно в контексте релаксации и сна.
    \item \textbf{Альфа-ритм:}
    Частотный диапазон от 8 до 12 Гц. Альфа-ритм является наиболее изученным и связан с состоянием спокойного бодрствования, расслабления и отсутствия внешних стимулов. Он наиболее выражен в затылочных областях и уменьшается при возникновении внимания или активации.
    \item \textbf{Бета-ритм:}
    Частотный диапазон от 12 до 30 Гц. Бета-ритм связан с активным состоянием бодрствования, когнитивной нагрузкой, вниманием и обработкой информации. Он отражает активацию корковых зон, участвующих в высших когнитивных функциях.
    \item \textbf{Гамма-ритм:}
    Частотный диапазон от 30 до 45 Гц.. Гамма-ритм связан с высокоуровневыми когнитивными процессами, такими как восприятие, внимание, обработка сложной информации и интеграция сенсорных данных. В последние годы исследования указывают на важную роль гамма-активности в когнитивных и эмоциональных процессах.
\end{enumerate}

В контексте поставленной задачи основное внимание уделяется альфа- и бета-диапазонам ЭЭГ-активности. В связи с этим предварительно обработанные данные подвергаются дополнительной полосовой фильтрации, ограничивающей спектр сигналов соответствующими частотными диапазонами. Подробности реализации процедуры фильтрации был приведены в разделе \ref{sec:freq-filtr}.


\section{Разложение на независимые компоненты}
После предварительной обработки и удаления артефактов следующим этапом анализа является разложение ЭЭГ-сигналов на независимые компоненты с использованием метода анализа независимых компонент (АНК), математическое описание которого было подробно рассмотрено в разделе \ref{sec:ica}. Данная процедура применяется отдельно для каждого из предварительно выделенных частотных диапазонов и направлена на декомпозицию сигналов на пространственно стабильные, статистически независимые компоненты, отражающие различные источники нейрональной активности. Такой подход позволяет выявить функционально обособленные нейрональные процессы, проявляющиеся в соответствующих частотных диапазонах, и служит основой для последующей локализации источников этих компонентов в коре головного мозга.

\section{Решение обратной задачи}

После выделения независимых компонент из ЭЭГ-сигналов с помощью ICA, следующим этапом анализа является решение обратной задачи. Ее суть заключается в определении источников мозговой активности на основании сигналов, зарегистрированных с поверхности головы. Поскольку данная задача математически некорректна (существует бесконечное множество решений), применяется регуляризация, обеспечивающая устойчивость решения. В настоящем исследовании использован подход, основанный на минимизации нормы распределения источников, с последующим нормированием по оценке шума, что позволяет получить значения, пропорциональные вероятности активации каждого участка.

Для получения решения обратной задачи применяется анатомическая модель мозга и информация о расположении регистрирующих электродов. Такие данные позволяют сформировать оператор перехода от источников к наблюдаемым данным и, соответственно, использовать его для восстановления пространственного распределения активности.

Математическая формулировка прямой задачи ЭЭГ имеет вид:

\begin{equation}
\mathbf{X}(t) = \mathbf{G} \cdot \mathbf{J}(t) + \boldsymbol{\varepsilon}(t),
\label{eq:forward}
\end{equation}

где:
\begin{itemize}
    \item \(\mathbf{X}(t) \in \mathbb{R}^{M}\) — вектор потенциалов, измеренных на \(M\) электродах в момент времени \(t\),
    \item \(\mathbf{G} \in \mathbb{R}^{M \times N}\) — матрица перехода (forward model), описывающая вклад каждого источника в каждый электрод,
    \item \(\mathbf{J}(t) \in \mathbb{R}^{N}\) — вектор токов в \(N\) источниках,
    \item \(\boldsymbol{\varepsilon}(t)\) — вектор шума наблюдений.
\end{itemize}

Обратная задача заключается в оценке \(\mathbf{J}(t)\) по \(\mathbf{X}(t)\). Один из подходов — метод минимальной нормы:

\begin{equation}
\hat{\mathbf{J}}(t) = \mathbf{W} \cdot \mathbf{X}(t),
\label{eq:inverse}
\end{equation}

где:
\begin{itemize}
    \item \(\hat{\mathbf{J}}(t)\) — оценка распределения источников в момент времени \(t\),
    \item \(\mathbf{W}\) — инверсионный (обратный) оператор.
\end{itemize}

Инверсионный оператор \(\mathbf{W}\) может быть вычислен следующим образом:

\begin{equation}
\mathbf{W} = \mathbf{R} \cdot \mathbf{G}^T \cdot \left( \mathbf{G} \cdot \mathbf{R} \cdot \mathbf{G}^T + \lambda^2 \cdot \mathbf{C} \right)^{-1},
\label{eq:inverse_operator}
\end{equation}

где:
\begin{itemize}
    \item \(\mathbf{R}\) — априорная ковариационная матрица источников,
    \item \(\mathbf{C}\) — ковариационная матрица шума,
    \item \(\lambda^2 = \frac{1}{\text{SNR}^2}\) — параметр регуляризации, выраженный через отношение сигнал/шум (SNR).
\end{itemize}


Примененный алгоритм делится на следующие этапы:

\subsection{Получение инверсионного оператора}


Процесс расчёта инверсионного оператора включает несколько ключевых этапов:

\begin{enumerate}
    \item \textbf{Формирование модели головы и решения задачи распространения электрических потенциалов.} На этом этапе создаётся модель геометрии головы, учитывающая разные ткани (например, кору, жидкость и кости). Далее строится решение задачи распространения потенциалов от источников в мозгу к электродам на поверхности головы. Математически это может быть выражено как:

    \begin{equation}
    \mathbf{V} = \mathbf{G} \cdot \mathbf{S},
    \end{equation}
    где:
    \begin{itemize}
        \item \(\mathbf{V}\) — измеренные потенциалы на электродах,
        \item \(\mathbf{G}\) — матрица распространения, которая описывает, как активность каждого источника влияет на электродные сигналы,
        \item \(\mathbf{S}\) — вектор источников активности в мозге.
    \end{itemize}
    
    \item \textbf{Дискретизация возможных источников активности.} На этом этапе создаётся сетка точек, в которых могут находиться источники активности. Эти точки представляют собой априорные места активности в коре головного мозга. Для описания источников используется сетка с расположением \(N_{\text{src}}\) точек. Это позволяет представить распределение источников как вектор:

    \begin{equation}
    \mathbf{S} = \left[ S_1, S_2, \dots, S_{N_{\text{src}}} \right]^T,
    \end{equation}
    где каждый элемент \(S_i\) — это активность источника в точке \(i\).

    \item \textbf{Построение матрицы прямой задачи.} На основе геометрической модели и дискретных точек источников создаётся матрица, которая описывает, как активность каждого источника влияет на измеренные сигналы с каждого электрода. Эта матрица \(\mathbf{G}\) имеет размерность \([N_{\text{elec}} \times N_{\text{src}}]\), где \(N_{\text{elec}}\) — количество электродов, а \(N_{\text{src}}\) — количество точек источников.

    \begin{equation}
    \mathbf{G} = \begin{bmatrix}
    g_{11} & g_{12} & \dots & g_{1N_{\text{src}}} \\
    g_{21} & g_{22} & \dots & g_{2N_{\text{src}}} \\
    \vdots & \vdots & \ddots & \vdots \\
    g_{N_{\text{elec}}1} & g_{N_{\text{elec}}2} & \dots & g_{N_{\text{elec}}N_{\text{src}}}
    \end{bmatrix},
    \end{equation}
    где \(g_{ij}\) — это вклад источника \(S_j\) в измерение на электроде \(i\).

    \item \textbf{Формирование шумовой ковариации.} Здесь строится ковариационная матрица, которая описывает свойства шума в измерениях. Эта матрица имеет вид:

    \begin{equation}
    \mathbf{C} = \begin{bmatrix}
    \sigma_1^2 & \cdots & \cdots & 0 \\
    \vdots & \ddots & \vdots & \vdots \\
    0 & \cdots & \cdots & \sigma_{N_{\text{elec}}}^2
    \end{bmatrix},
    \end{equation}
    где \(\sigma_i^2\) — дисперсия шума на \(i\)-м электроде.

    \item \textbf{Построение инверсионного оператора.} На основе информации о прямой задаче и ковариации шума рассчитывается инверсионный оператор, который позволит восстановить источники активности на основе наблюдаемых сигналов. Формула для инверсионного оператора, как уже было представлено ранее, имеет вид:

    \begin{equation}
    \mathbf{W} = \mathbf{R} \cdot \mathbf{G}^T \cdot \left( \mathbf{G} \cdot \mathbf{R} \cdot \mathbf{G}^T + \lambda^2 \cdot \mathbf{C} \right)^{-1},
    \end{equation}
    где:
    \begin{itemize}
        \item \(\mathbf{W}\) — инверсионный оператор,
        \item \(\mathbf{G}\) — матрица прямой задачи,
        \item \(\mathbf{R}\) — ковариационная матрица источников,
        \item \(\mathbf{C}\) — ковариационная матрица шума,
        \item \(\lambda^2 = \frac{1}{\text{SNR}^2}\) — параметр регуляризации.
    \end{itemize}
\end{enumerate}

После получения инверсионного оператора он сохраняется для дальнейшего использования в процессе локализации активности источников. Этот оператор играет ключевую роль в обратной задаче, позволяя преобразовывать наблюдаемые данные в распределение активности по источникам в мозге.


\subsection{Формирование сигналов независимых компонент}

Временной сигнал \(s_k(t)\) каждой независимой компоненты \(k\) восстанавливается из исходных каналов следующим образом:

\begin{equation}
s_k(t) = \sum_{i=1}^{M} a_{ik} \cdot x_i(t),
\label{eq:ica_signal}
\end{equation}

где:
\begin{itemize}
    \item \(s_k(t)\) — временной сигнал \(k\)-й независимой компоненты,
    \item \(a_{ik}\) — коэффициент вклада \(i\)-го канала в \(k\)-ю компоненту (из матрицы ICA-разложения),
    \item \(x_i(t)\) — сигнал на \(i\)-м канале в момент времени \(t\),
    \item \(M\) — число каналов ЭЭГ.
\end{itemize}

\subsection{Сегментация сигнала на интервалы}

Временной ряд компоненты разбивается на сегменты фиксированной длины:

\begin{equation}
\text{Epoch}_j = s_k(t_j : t_{j+1}), \quad j = 1, \dots, P,
\label{eq:segmentation}
\end{equation}

где:
\begin{itemize}
    \item \(\text{Epoch}_j\) — \(j\)-й сегмент сигнала,
    \item \(t_j, t_{j+1}\) — начальный и конечный моменты времени для сегмента,
    \item \(P\) — общее количество сегментов.
\end{itemize}

\subsection{Решение обратной задачи для каждого сегмента}

Для каждого сегмента \(\text{Epoch}_j\) вычисляется пространственное распределение источников:

\begin{equation}
\hat{\mathbf{J}}_j = \mathbf{W} \cdot \text{Epoch}_j,
\label{eq:epoch_inverse}
\end{equation}

где:
\begin{itemize}
    \item \(\hat{\mathbf{J}}_j \in \mathbb{R}^{N \times T}\) — матрица активности \(N\) источников по \(T\) временным точкам,
    \item \(\mathbf{W} \in \mathbb{R}^{N \times M}\) — инверсионный оператор,
    \item \(\text{Epoch}_j \in \mathbb{R}^{M \times T}\) — матрица сегмента сигнала на \(M\) каналах.
\end{itemize}

\subsection{Усреднение активности по времени}

Для каждого источника \(n\) вычисляется средняя активность по времени:

\begin{equation}
\bar{J}_n = \frac{1}{T} \sum_{t=1}^{T} \hat{J}_n(t),
\label{eq:mean_activity}
\end{equation}

где:
\begin{itemize}
    \item \(\bar{J}_n\) — средняя активность \(n\)-го источника,
    \item \(\hat{J}_n(t)\) — активность источника \(n\) в момент времени \(t\),
    \item \(T\) — количество временных точек в сегменте.
\end{itemize}




\endinput                                     % Вторая глава
\chapter{Представление результатов}
Анализ результатов локализации источников мозговой активности осуществляется на основе визуализации инвертированных топографических карт, полученных из ЭЭГ данных для двух различных экспериментальных условий. Визуализация производится как в виде пространственного распределения активности на поверхности коры головного мозга, так и в проекции на анатомические области.

Визуальные карты активности строятся по результатам решения обратной задачи для каждой независимой компоненты, извлечённой из предварительно обработанных сигналов. Эти карты отражают локализацию предполагаемых нейронных источников, ответственых за генерацию наблюдаемых паттернов ЭЭГ-активности. Инвертированные карты представлены в проекции на стандартную мозговую модель, что обеспечивает сравнимость данных между субъектами и экспериментами.

Для более детализированного анализа пространственное распределение активности дополнительно декомпозируется по анатомическим областям мозга, с опорой на Бродмановскую классификацию коры. Каждой компоненте сопоставляется набор Бродмановских полей, в которых наблюдается наибольшая выраженность источниковой активности. Такая декомпозиция позволяет интерпретировать полученные данные в терминах функциональной организации коры головного мозга.

Кроме того, компоненты анализируются в контексте их частотной принадлежности, благодаря предварительной фильтрации сигналов по заданным диапазонам частот, что, в свою очередь, даёт основания для гипотез о функциональном значении наблюдаемых паттернов.

На основании полученных распределений будет проведена количественная оценка доли вклада каждого из Бродмановских полей в генерацию независимых компонентов. Это позволит выявить специфические области, наиболее активные при различных экспериментальных условиях и в различных частотных диапазонах, а также провести сравнение между условиями (например, «музыка» против «тишины»), что может пролить свет на механизмы нейронной активности в этих состояниях.

Таким образом, представленные результаты сочетают в себе как качественную визуальную интерпретацию активности, так и количественную статистическую оценку пространственного распределения нейронных источников в анатомическом и функциональном контексте.

\endinput                                     % Третья глава
\chapter*{Заключение}
\addcontentsline{toc}{chapter}{Заключение}
В ходе выполненного исследования был разработан алгоритм, решающий задачу локализации областей коры головного мозга с пониженной нейронной активностью на основе ЭЭГ данных, полученных в условиях двух различных когнитивных состояний: во время прослушивания музыкального фрагмента и в состоянии тишины. Алгоритм включает этапы предобработки сигналов, извлечения независимых компонент, выделения частотных диапазонов, решения обратной задачи и пространственной проекции результатов на анатомические области коры.

На текущем этапе завершена реализация технической части алгоритма, позволяющей проводить пространственную локализацию источников активности и представлять полученные данные как в визуальной, так и в количественной форме. В дальнейшем планируется проведение статистического анализа полученных результатов с целью выявления устойчивых закономерностей в распределении нейронной активности по анатомическим областям в зависимости от когнитивного состояния.

Таким образом, созданная методика закладывает основу для последующего изучения механизмов нейронной активности и особенностей её распределения в условиях различной сенсорной нагрузки. Полученные результаты могут иметь значение для фундаментальных нейронаук и прикладных задач, включая нейродиагностику и когнитивную нейрореабилитацию.

\endinput                                % Заключение

\begin{thebibliography}{9}
 \bibitem{label1} С. Гонсалес, Р. Вудс "Цифровая обработка изображений". - Москва. - 2012.
\bibitem{label2} Э. Марк, Роберт Ф, "Магнитно-резонансная томография: физические принципы и дизайн последовательности". - Нью-Йорк. - 1999.
\bibitem{mne} MNE-Python documentation: Filtering and preprocessing. \href{https://mne.tools/stable/}{Ссылка}.
\newline
https://mne.tools/stable/

\bibitem{oppenheim} Oppenheim, A. V., Schafer, R. W., \& Buck, J. R. (1999). \textit{Discrete-Time Signal Processing}. Prentice Hall.
\end{thebibliography}
\endinput

\end{document}