\chapter*{Заключение}
\addcontentsline{toc}{chapter}{Заключение}
\section*{Анализ данных о средних значений процента неактивной области}
По результатам данных из \hyperref[tab:table]{Таблицы 1} можно сделать вывод, что наименее равномерное распределение мощностей сигнала в частотной области определенного канала приходится на delta- и gamma- ритмы. Это отражает нерегулярное расределенеие, что связано с локальными различиями и вовлеченностью отдельных зон мозга для данных ритмов. На примере alpha- и theta- каналов можно наблюдать отсутствие локализированной активности.
\section*{Анализ видеороликов}
Вследствие анализа видеороликов, было выявлено, что наименее активной зоной является центральная зона головного мозга, отвечающая за моторную активность. Это может свидетельствовать о переходе мозга в более спокойное состояние, вследсвие расслабления, или об отсутсвии дивгательной активности.

\endinput