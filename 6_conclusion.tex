\chapter*{Заключение}
\addcontentsline{toc}{chapter}{Заключение}
В ходе выполненного исследования был разработан алгоритм, решающий задачу локализации областей коры головного мозга с пониженной нейронной активностью на основе ЭЭГ данных, полученных в условиях двух различных когнитивных состояний: во время прослушивания музыкального фрагмента и в состоянии тишины. Алгоритм включает этапы предобработки сигналов, извлечения независимых компонент, выделения частотных диапазонов, решения обратной задачи и пространственной проекции результатов на анатомические области коры.

На текущем этапе завершена реализация технической части алгоритма, позволяющей проводить пространственную локализацию источников активности и представлять полученные данные как в визуальной, так и в количественной форме. В дальнейшем планируется проведение статистического анализа полученных результатов с целью выявления устойчивых закономерностей в распределении нейронной активности по анатомическим областям в зависимости от когнитивного состояния.

Таким образом, созданная методика закладывает основу для последующего изучения механизмов нейронной активности и особенностей её распределения в условиях различной сенсорной нагрузки. Полученные результаты могут иметь значение для фундаментальных нейронаук и прикладных задач, включая нейродиагностику и когнитивную нейрореабилитацию.

\endinput