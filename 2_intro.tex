\chapter*{Введение}
\addcontentsline{toc}{chapter}{Введение}
\label{ch:intro}
Электроэнцефалограмма (ЭЭГ) — метод регистрации электрической активности головного мозга, который применяется для диагностики неврологических расстройств, мониторинга функционального состояния мозга и научных исследований в области нейрофизиологии. В медицинской практике она используется для диагностики эпилепсии, расстройств сна, оценивания состояния мозга после травм, инсультов и других паталогических состояний. В нейрофизиологии и когнитивных науках ЭЭГ применяется для изучения функций мозга, таких как внимание, восприятие и память.

ЭЭГ позволяет получить информацию о временной динамике и распределении электрической активности мозга с помощью электродов, размещенных на поверхности головы. Сигналы, записываемые в процессе ЭЭГ, представляют собой колебания потенциалов, возникающих в результате активности нейронных связей. Эти колебания могут быть классифицированы по частотным диапазонам, которые соответствуют различным состояниям мозга и когнитивным процессам. Основные частотные диапазоны, выделяемые в ЭЭГ-сигналах, включают:
\begin{enumerate}
    \item Дельта-ритм (Delta):
    Частотный диапазон от 0 до 4 Гц. Дельта-ритм связан с глубоким сном, состоянием покоя и восстановления. В норме он наблюдается у здоровых людей только в состоянии глубокого сна, но у пациентов с эпилепсией или органическими поражениями мозга может проявляться в состоянии бодрствования.
    \item Тета-ритм (Theta):
     Частотный диапазон от 4 до 8 Гц. Тета-ритм ассоциируется с состоянием расслабления, медитацией, сном и эмоциональными процессами. Он также связан с памятью и когнитивным функционированием, особенно в контексте релаксации и сна.
    \item Альфа-ритм (Alpha):
    Частотный диапазон от 8 до 12 Гц. Альфа-ритм является наиболее изученным и связан с состоянием спокойного бодрствования, расслабления и отсутствия внешних стимулов. Он наиболее выражен в затылочных областях и уменьшается при возникновении внимания или активации.
    \item Бета-ритм (Beta):
    Частотный диапазон от 12 до 30 Гц. Бета-ритм связан с активным состоянием бодрствования, когнитивной нагрузкой, вниманием и обработкой информации. Он отражает активацию корковых зон, участвующих в высших когнитивных функциях.
    \item Гамма-ритм (Gamma):
    Частотный диапазон от 30 до 45 Гц.. Гамма-ритм связан с высокоуровневыми когнитивными процессами, такими как восприятие, внимание, обработка сложной информации и интеграция сенсорных данных. В последние годы исследования указывают на важную роль гамма-активности в когнитивных и эмоциональных процессах.
\end{enumerate}

В классических парадигмах анализа ЭЭГ основное внимание уделяется активным зонам — тем участкам коры, где наблюдается повышение мощности в определённых диапазонах частот. Предполагается, что именно эти зоны вовлечены в текущую когнитивную или сенсорную задачу. Однако в настоящей работе ставится под сомнение эта традиционная интерпретация. Исходной гипотезой исследования является предположение о том, что наиболее важные для текущей задачи участки мозга могут проявляться на ЭЭГ не как активные, а наоборот — как участки с пониженной выраженностью сигналов, то есть как "неактивные зоны" в классическом смысле. Такая интерпретация может отражать усиление локального торможения, перераспределение внимания или подавление фона, и, следовательно, быть маркером функционального вовлечения.

Целью данной работы является разработка алгоритма для локализации участков коры головного мозга, демонстрирующих пониженную активность, и последующее сравнение этих участков в двух различных условиях: при прослушивании музыки и в состоянии тишины. Таким образом, работа направлена на поиск тех зон мозга, которые наиболее вовлечены в восприятие музыки или в пассивное состояние, вопреки традиционному акценту на максимумах ЭЭГ-сигнала.

Для достижения этой цели были поставлены следующие задачи:

\begin{enumerate}
    \item провести предварительную обработку ЭЭГ-данных, включая частотную фильтрацию и автоматическое удаление артефактов с помощью ICAlabel.
    \item выделить конкретные частотные диапазоны и провести повторное разложение на независимые компоненты (ICA).
    \item решить обратную задачу ЭЭГ, с акцентом на компоненты, демонстрирующие минимальную активность.
    \item визуализировать и проанализировать полученные карты неактивности в условиях "музыка" и "тишина".
\end{enumerate}

Настоящая работа включает в себя теоретическое обоснование применяемых методов, описание алгоритма обработки и анализа ЭЭГ-данных, а также обсуждение полученных результатов в контексте альтернативной интерпретации нейронной активности.

\endinput

%Электроэнцефалограмма (ЭЭГ) — метод регистрации электрической активности головного мозга, который применяется для диагностики неврологических расстройств, мониторинга функционального состояния мозга и научных исследований в области нейрофизиологии. В медицинской практике она используется для диагностики эпилепсии, расстройств сна, оценивания состояния мозга после травм, инсультов и других паталогических состояний. В нейрофизиологии и когнитивных науках ЭЭГ применяется для изучения функций мозга, таких как внимание, восприятие и память.

% ЭЭГ позволяет получить информацию о временной динамике и распределении электрической активности мозга с помощью электродов, размещенных на поверхности головы. Сигналы, записываемые в процессе ЭЭГ, представляют собой колебания потенциалов, возникающих в результате активности нейронных связей. Эти колебания могут быть классифицированы по частотным диапазонам, которые соответствуют различным состояниям мозга и когнитивным процессам. Основные частотные диапазоны, выделяемые в ЭЭГ-сигналах, включают:
% \begin{enumerate}
%     \item Дельта-ритм (Delta):
%     Частотный диапазон от 0 до 4 Гц. Дельта-ритм связан с глубоким сном, состоянием покоя и восстановления. В норме он наблюдается у здоровых людей только в состоянии глубокого сна, но у пациентов с эпилепсией или органическими поражениями мозга может проявляться в состоянии бодрствования.
%     \item Тета-ритм (Theta):
%      Частотный диапазон от 4 до 8 Гц. Тета-ритм ассоциируется с состоянием расслабления, медитацией, сном и эмоциональными процессами. Он также связан с памятью и когнитивным функционированием, особенно в контексте релаксации и сна.
%     \item Альфа-ритм (Alpha):
%     Частотный диапазон от 8 до 12 Гц. Альфа-ритм является наиболее изученным и связан с состоянием спокойного бодрствования, расслабления и отсутствия внешних стимулов. Он наиболее выражен в затылочных областях и уменьшается при возникновении внимания или активации.
%     \item Бета-ритм (Beta):
%     Частотный диапазон от 12 до 30 Гц. Бета-ритм связан с активным состоянием бодрствования, когнитивной нагрузкой, вниманием и обработкой информации. Он отражает активацию корковых зон, участвующих в высших когнитивных функциях.
%     \item Гамма-ритм (Gamma):
%     Частотный диапазон от 30 до 45 Гц.. Гамма-ритм связан с высокоуровневыми когнитивными процессами, такими как восприятие, внимание, обработка сложной информации и интеграция сенсорных данных. В последние годы исследования указывают на важную роль гамма-активности в когнитивных и эмоциональных процессах.
% \end{enumerate}
% Однако, кроме полезных сигналов, в данные ЭЭГ попадают различные помехи, которые необходимо устранить для корректного анализа. Одним из наиболее часто встречающихся типов помех являются низкочастотные дрейфы, которые обычно имеют частоту ниже 1 Гц и могут быть вызваны неравномерным дыханием, изменениями кровотока или артериального давления, а также изменениями температуры тела. Эти помехи проявляются в виде медленных изменений базовой линии сигнала. 
% Другой распространенный тип помех — мышечные артефакты, обусловленные активностью мышц головы, таких как жевательные мышцы или движения глаз, которые накладывают высокочастотные шумы на ЭЭГ-сигналы.

% После предварительной очистки и выделения интересующих частотных диапазонов, проводится разложение сигналов методом ICA. Полученные компоненты далее используются для пространственной локализации источников мозговой активности посредством решения обратной задачи ЭЭГ.

% Решение обратной задачи направлено на определение пространственного распределения активности в головном мозге по данным с поверхности черепа. Это фундаментально некорректная (недоопределённая) задача, однако современные методы, такие как LORETA, sLORETA или другие линейные модели, позволяют получить приближенные оценки. В данной работе решается инвертированная обратная задача, где целью является не выявление активных зон, а, наоборот, поиск участков, демонстрирующих наименьшую активность. Для этого рассчитывается среднее значение сигнала по каналам, после чего из максимального значения по каждому каналу вычитается это среднее. Такой подход позволяет визуализировать области, систематически демонстрирующие подавленную активность, что может быть важным маркером состояний подавления, «отключения» или перераспределения внимания.

% Выбор метода обратной задачи обусловлен необходимостью не просто анализировать временные характеристики ЭЭГ, но и получить топографическую карту корковых источников, что критично при сравнении функциональной архитектуры мозга между различными экспериментальными условиями.