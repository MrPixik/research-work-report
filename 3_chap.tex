\chapter{Предобработка ЭЭГ-данных}
\label{ch:chap1}
Предобработка ЭЭГ данных критически важна для дальнейшего анализа и интерпретации результатов. Она включает в себя несколько ключевых этапов, направленных на улучшение качества сигналов и снижение влияния помех. В настоящей главе представлен полный алгоритм предобработки данных, который был применен в данной работе, а также краткий теоретический экскурс по каждому из используемых методов.

\section{Частотная фильтрация}
\label{sec:freq-filtr}
ЭЭГ-сигналы подвержены различным высокочастотным и низкочастотным помехам, таким как шумы от электроприборов, мышечных артефактов и дыхания. Для удаления нежелательных частот применяют фильтрацию.

В частности, в данной программе была примененен высокочастотный КИХ-фильтр, построенный на основе окна Хэмминга, для удаления низкочастотных дрейфов.

В данном фильтре фильтрация временного ряда выполняется с помощью свёртки исходного сигнала \(x[n]\) с импульсной характеристикой фильтра \(h[k]\). Формула фильтрации имеет вид:

\begin{equation}
y[n] = \sum_{k=0}^{M} h[k] \cdot x[n-k], \quad n = M, \dots, N-M,
\label{eq:fir_filter}
\end{equation}

где:
\begin{itemize}
    \item \(y[n]\) — выходной (отфильтрованный) сигнал,
    \item \(x[n]\) — входной сигнал,
    \item \(h[k]\) — импульсная характеристика фильтра,
    \item \(M\) — длина фильтра,
    \item \(N\) — длина входного сигнала.
\end{itemize}

Для сглаживания коэффициентов фильтра используется окно Хэмминга, определяемое следующим образом:

\begin{equation}
w[k] = 0.54 - 0.46 \cos\left( \frac{2 \pi k}{M} \right), \quad k = 0, \dots, M.
\label{eq:hamming_window}
\end{equation}

Импульсная характеристика высокочастотного фильтра с частотой отсечки \(f_c\) задаётся как:

\begin{equation}
h[k] = w[k] \cdot \left( \delta[k] - \frac{\sin(2 \pi f_c (k - M/2))}{\pi (k - M/2)} \right), \quad k = 0, \dots, M,
\label{eq:filter_kernel}
\end{equation}

где:
\begin{itemize}
    \item \(\delta[k]\) — дельта-функция (единица при \(k = M/2\), иначе ноль),
    \item \(f_c\) — частота отсечки в долях от частоты дискретизации,
    \item \(w[k]\) — окно Хэмминга (см. формулу \eqref{eq:hamming_window}).
\end{itemize}

Выбор именно этого фильтра объясняется следующими соображениями. Во-первых, он не искажает форму сигнала, что важно для ЭЭГ, где даже небольшие сдвиги могут исказить результаты анализа. Во-вторых, окно Хэмминга помогает отсечь частоты вне нашего диапазона без сильного «размытия» полезной части сигнала. Также, применение КИХ-фильтров для предварительной фильтрации ЭЭГ является общепринятым подходом.

\section{Удаление мимических артефактов.}
\label{sec:ica}
ЭЭГ данные помимо высокочастотных и нихкочастотных помех содержат различные мимческие артефакты, частоты которых совпадают с частотами, которые несут полезную информацию о мозговой активности. Для удаленяия такого рода артефактов, данные предварительно разбиваются на компоненты посредством применения метода анализа независимых компонент (АНК). 

Математически модель АНК можно представить следующим образом:

\begin{equation}
\mathbf{X} = \mathbf{A} \mathbf{S},
\label{eq:ica_model}
\end{equation}

где:
\begin{itemize}
    \item \(\mathbf{X} \in \mathbb{R}^{n \times T}\) — матрица наблюдаемых сигналов ЭЭГ, где \(n\) — количество каналов, \(T\) — количество временных отсчётов;
    \item \(\mathbf{A} \in \mathbb{R}^{n \times n}\) — неизвестная матрица смешивания, описывающая вклад каждой компоненты в каждый канал;
    \item \(\mathbf{S} \in \mathbb{R}^{n \times T}\) — матрица скрытых независимых компонент, которые необходимо восстановить.
\end{itemize}

Цель АНК заключается в нахождении такой матрицы \(\mathbf{W}\), которая аппроксимирует обратную матрицу смешивания \(\mathbf{A}^{-1}\) и позволяет выделить независимые компоненты:

\begin{equation}
\mathbf{S} = \mathbf{W} \mathbf{X},
\label{eq:ica_unmixing}
\end{equation}

где:
\begin{itemize}
    \item \(\mathbf{W} \in \mathbb{R}^{n \times n}\) — матрица разведения (unmixing matrix), обучаемая на основе статистической независимости компонент;
    \item \(\mathbf{S}\) — восстановленные независимые источники сигнала.
\end{itemize}

Метод АНК основывается на предположении, что все компоненты \(\mathbf{S}\) являются статистически независимыми и неггауссовыми (за исключением, возможно, одной). Благодаря этому он эффективно разделяет смешанные сигналы и позволяет локализовать артефакты, не прибегая к априорной информации о природе источников.

После разложения сигналов на компоненты с помощью АНК производится автоматическая классификация компонент с использованием инструмента ICLabel. Он представляет собой обученную модель, которая относит полученные компоненты к различным классам (мозг, мышцы, глаза, шум и т.д.). После классификации, нежелательные компоненты удаляются.

Использование АНК в сочетании с ICLabel обеспечивает эффективное и автоматизированное удаление артефактов из ЭЭГ-данных, сохраняя при этом значимую нейронную информацию. Этот подход широко применяется в нейрофизиологических исследованиях и клинической практике для повышения качества анализа мозговой активности.

\endinput