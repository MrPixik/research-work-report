\chapter*{Аннотация}
\label{ch:annotation}
В данной работе рассматривается задача локализации зон коры головного мозга с пониженной нейронной активностью в условиях двух различных когнитивных состояний: при прослушивании музыкального фрагмента и в тишине. Целью исследования является разработка алгоритма, позволяющего выявлять и визуализировать участки мозга, демонстрирующие минимальную активность в каждом из условий. Предложенный алгоритм включает в себя этапы предобработки данных ЭЭГ (частотная фильтрация, удаление артефактов с использованием метода независимого компонентного анализа и автоматической классификации ICAlabel), повторную декомпозицию сигналов по частотным диапазонам, а также решение обратной задачи для пространственной локализации активности.
Особенностью подхода является ориентация на зоны с наименьшей, а не максимальной, активностью, что позволяет исследовать механизмы подавления или «отключения» отдельных участков коры в зависимости от сенсорного контекста.
\vspace*{-\baselineskip}
